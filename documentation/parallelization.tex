Il codice sequenziale di riferimento da parallelizzare per la fase di setup è il seguente, rispettivamente per la versione c++ e python:
\lstinputlisting[language=c++, firstline=22, lastline=28,label={lst:sequential-code-setup-omp}]{BloomFilter.cpp}
\lstinputlisting[language=python, firstline=33, lastline=39,label={lst:sequential-code-setup-joblib}]{../src/bloom_filter.py}

Nella fase sequenziale della configurazione, dopo un'iniziale fase di inizializzazione finalizzata al calcolo del valore
ottimale del vettore di bit e del numero di funzioni hash in base al valore di probabilità di falsi positivi,
si procede con l'indicizzazione nel vettore di bit per ciascuna email.

Il codice sequenziale di riferimento da parallelizzare per la fase di filter è il seguente, rispettivamente per la versione c++ e python:
\lstinputlisting[language=c++, firstline=54, lastline=60,label={lst:sequential-code-filter-omp}]{BloomFilter.cpp}
\lstinputlisting[language=python, firstline=58, lastline=66,label={lst:sequential-code-filter-joblib}]{../src/bloom_filter.py}

Nella fase sequenziale del processo di filtraggio, si inizializza una variabile di conteggio per i
falsi positivi, la quale sarà incrementata ogni volta che un'email viene identificata come spam.

\subsection{OpenMP}\label{subsec:omp}
OpenMP (Omp) è una libreria in linguaggio C progettata per consentire la parallelizzazione di funzioni e cicli for.
Nel contesto di questo lavoro, abbiamo adottato la funzione \texttt{omp parallel for} per parallelizzare le operazioni
di setup e filtraggio del BloomFilter.

Successivamente, abbiamo analizzato il tempo impiegato dalle funzioni di setup e filtraggio in relazione al numero
di processi utilizzati, confrontandolo con il tempo richiesto nella versione sequenziale.
\lstinputlisting[language=c++, firstline=30, lastline=47, label={lst:openmp-setup-code}]{BloomFilter.cpp}
Per la fase di setup, abbiamo parallelizzato l'indicizzazione nel vettore di bit per ogni email, utilizzando la direttiva \texttt{omp parallel for},
e la direttiva \texttt{omp critical} per garantire l'accesso esclusivo al vettore di bit, evitando così l'accesso concorrente.

Per quanto riguarda l'operazione di filtraggio, abbiamo ideato due diverse implementazioni.
\lstinputlisting[language=c++, firstline=62, lastline=74, label={lst:openmp-filter-code}]{BloomFilter.cpp}
La prima realizzazione prevede l'impiego della direttiva \texttt{omp parallel for} al fine di parallelizzare il processo
di verifica della presenza di un'email all'interno del BloomFilter.
In aggiunta, viene utilizzata la direttiva \texttt{omp atomic} per assicurare l'accesso esclusivo alla variabile
incrementale utilizzata per il conteggio dei falsi positivi.

\lstinputlisting[language=c++, firstline=76, lastline=90, label={lst:openmp-filter2-code}]{BloomFilter.cpp}
La seconda realizzazione, al contrario, impiega una variabile incrementale dedicata per ciascun thread, e la somma di
tali variabili al termine dell'operazione di filtraggio.
Inoltre, si fa uso della direttiva \texttt{omp atomic} per assicurare l'accesso esclusivo alla variabile
incrementale globale durante questa fase.
Verificheremo in seguito la differenza tra queste due implementazioni.

\subsection{Joblib}\label{subsec:joblib}
Joblib è una libreria Python ideata per abilitare la parallelizzazione di funzioni e cicli for.
La funzione \texttt{Parallel} prende come input sia il numero di processi da utilizzare che la funzione da parallelizzare.
Nell'ambito di questo studio, abbiamo sfruttato la funzione \texttt{Parallel} per parallelizzare sia le fasi di
preparazione (setup) che quelle di filtraggio del BloomFilter.

Successivamente, abbiamo analizzato il tempo richiesto dalle funzioni di setup e filtraggio in relazione al numero
di processi utilizzati, confrontandolo con il tempo necessario nella versione sequenziale.
\lstinputlisting[language=python, firstline=41, lastline=50,label={lst:joblib-code-setup}]{../src/bloom_filter.py}
Analogamente all'approccio con Omp, abbiamo parallelizzato l'indicizzazione nel vettore di bit per ciascuna email
mediante l'utilizzo della funzione \texttt{Parallel}, suddividendo il vettore di email in chunk corrispondenti al
numero di processi impiegati.
In aggiunta, il vettore di bit viene memorizzato in memoria attraverso l'utilizzo della funzione \texttt{memmap},
consentendo sia di condividerlo tra i vari processi che di conservare i risultati ottenuti in memoria.

\lstinputlisting[language=python, firstline=68, lastline=80,label={lst:joblib-code-filter}]{../src/bloom_filter.py}
Nella fase di filtraggio, abbiamo parallelizzato la verifica della presenza di un'email nel BloomFilter mediante
l'utilizzo della funzione \texttt{Parallel}.
Anche in questo caso, abbiamo diviso il vettore di email in chunk corrispondenti al numero di processi impiegati.

In seguito, abbiamo voluto esaminare l'impatto della dimensione del chunk anche al di là del numero di processi
utilizzati, al fine di valutare se un suo aumento potesse o meno offrire vantaggi in termini di tempo e di speedup.